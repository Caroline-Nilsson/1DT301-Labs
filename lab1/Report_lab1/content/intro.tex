In the process of working with the laboratory assignments we started by doing research about the AVR Assembly language and the STK600 in order to better understand how to solve the different assignments.
In each assignment we first created a pseudocode solution which we converted to flowchart diagrams, then it was rather simple to convert this into Assembly code. Common for all assignments is also that we have been using the simulations to confirm that the program is working and completing the correct tasks. 

When we tested the Assembly programs on hardware we had trouble with the LEDs outputting inversed values. For example the program for assignment 1, which is supposed light \texttt{LED2}, lit all LEDs except \texttt{LED2}. If we understood this correctly, this was due to the pull-up resistor being activated on \texttt{PORTB} which made the LEDs light when the corresponding bit on \texttt{PORTB} was 0 (as opposed to 1). We fixed this by changing the bit string values we wrote to the LEDs in each assignment.

Another small issue we encountered was that we had assumed that \texttt{PORTD} would be the port connected to the switches on the STK600, as this is the configuration described in the STK600 User Guide. In the hardware we tested on though, the switches were connected to \texttt{PORTC}. To fix this we simply replaced all references of \texttt{PORTD}, \texttt{PIND} and \texttt{DDRD} to \texttt{PORTC}, \texttt{PINC} and \texttt{DDRC} in our Assembly code.
